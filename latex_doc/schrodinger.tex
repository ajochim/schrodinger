%% Generated by Sphinx.
\def\sphinxdocclass{report}
\documentclass[letterpaper,10pt,english]{sphinxmanual}
\ifdefined\pdfpxdimen
   \let\sphinxpxdimen\pdfpxdimen\else\newdimen\sphinxpxdimen
\fi \sphinxpxdimen=.75bp\relax

\usepackage[utf8]{inputenc}
\ifdefined\DeclareUnicodeCharacter
 \ifdefined\DeclareUnicodeCharacterAsOptional
  \DeclareUnicodeCharacter{"00A0}{\nobreakspace}
  \DeclareUnicodeCharacter{"2500}{\sphinxunichar{2500}}
  \DeclareUnicodeCharacter{"2502}{\sphinxunichar{2502}}
  \DeclareUnicodeCharacter{"2514}{\sphinxunichar{2514}}
  \DeclareUnicodeCharacter{"251C}{\sphinxunichar{251C}}
  \DeclareUnicodeCharacter{"2572}{\textbackslash}
 \else
  \DeclareUnicodeCharacter{00A0}{\nobreakspace}
  \DeclareUnicodeCharacter{2500}{\sphinxunichar{2500}}
  \DeclareUnicodeCharacter{2502}{\sphinxunichar{2502}}
  \DeclareUnicodeCharacter{2514}{\sphinxunichar{2514}}
  \DeclareUnicodeCharacter{251C}{\sphinxunichar{251C}}
  \DeclareUnicodeCharacter{2572}{\textbackslash}
 \fi
\fi
\usepackage{cmap}
\usepackage[T1]{fontenc}
\usepackage{amsmath,amssymb,amstext}
\usepackage{babel}
\usepackage{times}
\usepackage[Bjarne]{fncychap}
\usepackage[dontkeepoldnames]{sphinx}

\usepackage{geometry}

% Include hyperref last.
\usepackage{hyperref}
% Fix anchor placement for figures with captions.
\usepackage{hypcap}% it must be loaded after hyperref.
% Set up styles of URL: it should be placed after hyperref.
\urlstyle{same}

\addto\captionsenglish{\renewcommand{\figurename}{Fig.}}
\addto\captionsenglish{\renewcommand{\tablename}{Table}}
\addto\captionsenglish{\renewcommand{\literalblockname}{Listing}}

\addto\captionsenglish{\renewcommand{\literalblockcontinuedname}{continued from previous page}}
\addto\captionsenglish{\renewcommand{\literalblockcontinuesname}{continues on next page}}

\addto\extrasenglish{\def\pageautorefname{page}}

\setcounter{tocdepth}{1}



\title{schrodinger Documentation}
\date{Sep 13, 2018}
\release{1.0}
\author{Tammo van der Heide and Alexander Jochim}
\newcommand{\sphinxlogo}{\vbox{}}
\renewcommand{\releasename}{Release}
\makeindex

\begin{document}

\maketitle
\sphinxtableofcontents
\phantomsection\label{\detokenize{index::doc}}



\chapter{Introduction}
\label{\detokenize{index:schrodinger}}\label{\detokenize{index:introduction}}
A program that solves the 1d schrodinger equation. It reads
a file \sphinxstyleemphasis{schrodinger.inp} that specifies a given potential and calculates
energies, wavefunctions, expectation values and standard deviation. All data
is plotted afterwards and saved to \sphinxstyleemphasis{schrodinger.pdf.}


\chapter{Input and Output}
\label{\detokenize{index:input-and-output}}
The name of the input file must be called \sphinxstylestrong{schrodinger.inp}. By default
\sphinxstyleemphasis{schrodinger} assumes it is located in the same directory as itself.
Other directory can be accessed via the \sphinxstyleemphasis{-d} starting option. The input file
must have the following structure:

\fvset{hllines={, ,}}%
\begin{sphinxVerbatim}[commandchars=\\\{\}]
\PYG{l+m+mf}{4.0}           \PYG{c+c1}{\PYGZsh{} Mass}
\PYG{o}{\PYGZhy{}}\PYG{l+m+mf}{5.0} \PYG{l+m+mf}{5.0} \PYG{l+m+mi}{1999} \PYG{c+c1}{\PYGZsh{} xMin xMax nPoint}
\PYG{l+m+mi}{1} \PYG{l+m+mi}{5}           \PYG{c+c1}{\PYGZsh{} first and last eigenvalue to include in the output}
\PYG{n}{polynomial}    \PYG{c+c1}{\PYGZsh{} interpolation type}
\PYG{l+m+mi}{3}             \PYG{c+c1}{\PYGZsh{} \PYGZsh{} nr. of interpolation points and xy declarations}
\PYG{o}{\PYGZhy{}}\PYG{l+m+mf}{1.0} \PYG{l+m+mf}{0.5}
\PYG{l+m+mf}{0.0} \PYG{l+m+mf}{0.0}
\PYG{l+m+mf}{1.0} \PYG{l+m+mf}{0.5}
\end{sphinxVerbatim}

You can use as many interpolation points as you want. The number has to fit
the number of xy declarations afterwards. Possible interpolation types are
\sphinxstyleemphasis{linear}, \sphinxstyleemphasis{csplines} and \sphinxstyleemphasis{polynomial}. \sphinxstyleemphasis{nPoints} is the number of discrete
points that are used for calculating.

All calculations are saved in the following files. They are used to plot
the data and can be used for further work:
\begin{itemize}
\item {} 
\sphinxstyleemphasis{potential.dat}

interpolated potential in XY-Format:

\fvset{hllines={, ,}}%
\begin{sphinxVerbatim}[commandchars=\\\{\}]
\PYG{n}{x1} \PYG{n}{V}\PYG{p}{(}\PYG{n}{x1}\PYG{p}{)}
\PYG{n}{x2} \PYG{n}{V}\PYG{p}{(}\PYG{n}{x2}\PYG{p}{)}
\PYG{p}{:}    \PYG{p}{:}
\end{sphinxVerbatim}

\item {} 
\sphinxstyleemphasis{energies.dat}

calculated eigenvalues

\fvset{hllines={, ,}}%
\begin{sphinxVerbatim}[commandchars=\\\{\}]
\PYG{n}{E1}
\PYG{n}{E2}
\PYG{n}{E3}
\PYG{p}{:}
\end{sphinxVerbatim}

\item {} 
\sphinxstyleemphasis{wavefuncs.dat}

calculated wavefunctions in NXY-Format

\fvset{hllines={, ,}}%
\begin{sphinxVerbatim}[commandchars=\\\{\}]
\PYG{n}{x1} \PYG{n}{wf1}\PYG{p}{(}\PYG{n}{x1}\PYG{p}{)} \PYG{n}{wf2}\PYG{p}{(}\PYG{n}{x1}\PYG{p}{)} \PYG{n}{wf3}\PYG{p}{(}\PYG{n}{x1}\PYG{p}{)} \PYG{o}{.}\PYG{o}{.}\PYG{o}{.}
\PYG{n}{x2} \PYG{n}{wf1}\PYG{p}{(}\PYG{n}{x2}\PYG{p}{)} \PYG{n}{wf2}\PYG{p}{(}\PYG{n}{x2}\PYG{p}{)} \PYG{n}{wf3}\PYG{p}{(}\PYG{n}{x2}\PYG{p}{)} \PYG{o}{.}\PYG{o}{.}\PYG{o}{.}
\PYG{p}{:}
\end{sphinxVerbatim}

\item {} 
\sphinxstyleemphasis{expvalues.dat}

expectation values and standard deviation

\fvset{hllines={, ,}}%
\begin{sphinxVerbatim}[commandchars=\\\{\}]
\PYG{n}{exp\PYGZus{}val1} \PYG{n}{st\PYGZus{}dev1}
\PYG{n}{exp\PYGZus{}val2} \PYG{n}{st\PYGZus{}dev2}
\PYG{p}{:}
\end{sphinxVerbatim}

\end{itemize}


\chapter{Starting Options}
\label{\detokenize{index:starting-options}}
Starting options (optional parameters) for the main module \sphinxstyleemphasis{schrodinger.py}.
Use the long form as -\sphinxstyleemphasis{-name} or use the short version showed below.


\section{Optional Parameters}
\label{\detokenize{index:optional-parameters}}\begin{itemize}
\item {} 
directory: -d {[}path{]}

Used to specify the path of the input file \sphinxstyleemphasis{schrodinger.inp}

\item {} 
split: -s

Splitting the wavefunctions, expectation values and standard deviations in
the plot for a better view.

\item {} 
stretch: -st {[}float{]}

Multiplies the wavefunctions with a factor for a better view.

\item {} 
markersize: -m {[}float{]}

Changes the markersize of the expectation values and standard deviation.

\end{itemize}


\chapter{Modules}
\label{\detokenize{index:modules}}

\section{schrodinger.py}
\label{\detokenize{index:schrodinger-py}}
Main module that can be used with the starting options (optional parameters)
above.


\section{schrodinger\_io.py}
\label{\detokenize{index:schrodinger-io-py}}\label{\detokenize{index:module-schrodinger_io}}\index{schrodinger\_io (module)}
Module that reads the input data, processed by the schrodinger solver
\index{output() (in module schrodinger\_io)}

\begin{fulllineitems}
\phantomsection\label{\detokenize{index:schrodinger_io.output}}\pysiglinewithargsret{\sphinxcode{schrodinger\_io.}\sphinxbfcode{output}}{\emph{data}}{}
Write calculated data to files
\begin{quote}\begin{description}
\item[{Parameters}] \leavevmode
\sphinxstyleliteralstrong{data} (\sphinxstyleliteralemphasis{dict}) \textendash{} dictionary with calculated data from solve1d

\end{description}\end{quote}

\end{fulllineitems}

\index{read\_input() (in module schrodinger\_io)}

\begin{fulllineitems}
\phantomsection\label{\detokenize{index:schrodinger_io.read_input}}\pysiglinewithargsret{\sphinxcode{schrodinger\_io.}\sphinxbfcode{read\_input}}{\emph{file}}{}
Read given parameters and potential data of ‘schrodinger.inp’
\begin{quote}\begin{description}
\item[{Parameters}] \leavevmode
\sphinxstyleliteralstrong{file} (\sphinxstyleliteralemphasis{str}) \textendash{} path to input file schrodinger.inp

\item[{Returns}] \leavevmode
obtained input from ‘file’

\item[{Return type}] \leavevmode
obtained\_input (dict)

\item[{Raises}] \leavevmode
\sphinxcode{OSError} \textendash{} if input file is not present or has wrong permissions/name

\end{description}\end{quote}

\end{fulllineitems}



\section{schrodinger\_solver.py}
\label{\detokenize{index:module-schrodinger_solver}}\label{\detokenize{index:schrodinger-solver-py}}\index{schrodinger\_solver (module)}
Module that solves the onedimensional Schrodinger equation for arbitrary potentials
\index{interpolate() (in module schrodinger\_solver)}

\begin{fulllineitems}
\phantomsection\label{\detokenize{index:schrodinger_solver.interpolate}}\pysiglinewithargsret{\sphinxcode{schrodinger\_solver.}\sphinxbfcode{interpolate}}{\emph{obtained\_input}}{}
Interpolation routine, that interpolates the given potential data with
a specified interpolation method
\begin{quote}\begin{description}
\item[{Parameters}] \leavevmode
\sphinxstyleliteralstrong{obtained\_input} (\sphinxstyleliteralemphasis{dict}) \textendash{} obtained input of schrodinger\_io.read\_input()

\item[{Returns}] \leavevmode
contains interpolated potential

\item[{Return type}] \leavevmode
interppot (numpy array)

\end{description}\end{quote}

\end{fulllineitems}

\index{solve1d() (in module schrodinger\_solver)}

\begin{fulllineitems}
\phantomsection\label{\detokenize{index:schrodinger_solver.solve1d}}\pysiglinewithargsret{\sphinxcode{schrodinger\_solver.}\sphinxbfcode{solve1d}}{\emph{obtained\_input}, \emph{pot}}{}
Solves the discretized 1D schrodinger equation
\begin{quote}\begin{description}
\item[{Parameters}] \leavevmode\begin{itemize}
\item {} 
\sphinxstyleliteralstrong{obtained\_input} (\sphinxstyleliteralemphasis{dict}) \textendash{} obtained input of schrodinger\_io.read\_input()

\item {} 
\sphinxstyleliteralstrong{pot} (\sphinxstyleliteralemphasis{numpy array}) \textendash{} contains potential

\end{itemize}

\item[{Returns}] \leavevmode
calculated data; potential, energies, wavefuncs, expvalues

\item[{Return type}] \leavevmode
data (dict)

\end{description}\end{quote}

\end{fulllineitems}



\section{schrodinger\_visualize.py}
\label{\detokenize{index:module-schrodinger_visualize}}\label{\detokenize{index:schrodinger-visualize-py}}\index{schrodinger\_visualize (module)}
Visualization module for the schrodinger\_solver
\index{show() (in module schrodinger\_visualize)}

\begin{fulllineitems}
\phantomsection\label{\detokenize{index:schrodinger_visualize.show}}\pysiglinewithargsret{\sphinxcode{schrodinger\_visualize.}\sphinxbfcode{show}}{\emph{stretchfactor=1}, \emph{split=False}, \emph{markersize=10}}{}
Visualizes the output of the solver function and saves to schrodinger.pdf
\begin{quote}\begin{description}
\item[{Parameters}] \leavevmode\begin{itemize}
\item {} 
\sphinxstyleliteralstrong{stretchfactor} (\sphinxstyleliteralemphasis{float}) \textendash{} stretches the wavefunctions in y-direction

\item {} 
\sphinxstyleliteralstrong{split} (\sphinxstyleliteralemphasis{bool}) \textendash{} creates offset to wavefunction, expectationsvalues

\item {} 
\sphinxstyleliteralstrong{standard deviation if set to 'True'} (\sphinxstyleliteralemphasis{and}) \textendash{} 

\item {} 
\sphinxstyleliteralstrong{markersize} (\sphinxstyleliteralemphasis{float}) \textendash{} changes markersize of expectations values and standard deviation

\end{itemize}

\item[{Raises}] \leavevmode
\sphinxcode{OSError} \textendash{} if input file is not present or has wrong permissions/name

\end{description}\end{quote}

\end{fulllineitems}



\section{test\_schrodinger\_solver.py}
\label{\detokenize{index:test-schrodinger-solver-py}}\label{\detokenize{index:module-test_schrodinger_solver}}\index{test\_schrodinger\_solver (module)}
Contains routines to test the solvers modules
\index{test\_compare() (in module test\_schrodinger\_solver)}

\begin{fulllineitems}
\phantomsection\label{\detokenize{index:test_schrodinger_solver.test_compare}}\pysiglinewithargsret{\sphinxcode{test\_schrodinger\_solver.}\sphinxbfcode{test\_compare}}{\emph{testname\_compare}}{}
Tests the constant functioning of the code by comparing previously
calculated energy-levels and potential data with the current output of the solver
\begin{quote}\begin{description}
\item[{Parameters}] \leavevmode
\sphinxstyleliteralstrong{testname\_compare} (\sphinxstyleliteralemphasis{str}) \textendash{} to create path of reference and input files

\end{description}\end{quote}
\begin{description}
\item[{Asserting:}] \leavevmode
test\_compare\_assert (bool): if ‘True’, test passes

\end{description}

\end{fulllineitems}

\index{test\_energies() (in module test\_schrodinger\_solver)}

\begin{fulllineitems}
\phantomsection\label{\detokenize{index:test_schrodinger_solver.test_energies}}\pysiglinewithargsret{\sphinxcode{test\_schrodinger\_solver.}\sphinxbfcode{test\_energies}}{\emph{testname\_energie}}{}
Tests the energy-levels of schrodinger\_solver by comparing thoose
with exact results or groundstates calculated on paper
\begin{quote}\begin{description}
\item[{Parameters}] \leavevmode\begin{itemize}
\item {} 
\sphinxstyleliteralstrong{testname\_energie} (\sphinxstyleliteralemphasis{str}) \textendash{} to create path of reference and input files

\item {} 
\sphinxstyleliteralstrong{energie data/parameters} (\sphinxstyleliteralemphasis{containing}) \textendash{} 

\end{itemize}

\end{description}\end{quote}
\begin{description}
\item[{Asserting:}] \leavevmode
test\_energies\_assert (bool): if ‘True’, test passes

\end{description}

\end{fulllineitems}

\index{test\_interpolation() (in module test\_schrodinger\_solver)}

\begin{fulllineitems}
\phantomsection\label{\detokenize{index:test_schrodinger_solver.test_interpolation}}\pysiglinewithargsret{\sphinxcode{test\_schrodinger\_solver.}\sphinxbfcode{test\_interpolation}}{\emph{testname\_interp}}{}
Tests the interpolation of schrodinger\_solver by comparing the
interpolated potential with the given XY data
\begin{quote}\begin{description}
\item[{Parameters}] \leavevmode\begin{itemize}
\item {} 
\sphinxstyleliteralstrong{testname\_interp} (\sphinxstyleliteralemphasis{str}) \textendash{} to create path of reference and input files

\item {} 
\sphinxstyleliteralstrong{potential data/parameters} (\sphinxstyleliteralemphasis{containing}) \textendash{} 

\end{itemize}

\end{description}\end{quote}
\begin{description}
\item[{Asserting:}] \leavevmode
test\_interp (bool): if ‘True’, test passes

\end{description}

\end{fulllineitems}



\chapter{Scientific Notes}
\label{\detokenize{index:scientific-notes}}
All calculations are numeric! The potential defined by discrete reference
points inside the input file is interpolated using numerical algorithms
(e.g., finite differences and integrals as Riemann sums).
The constructions of a tridiagonal matrix allows solving the time independent
schrodinger equation at discrete points as an eigenvalue problem. This
results in inaccuracies both due to discretization errors and due to rounding
errors in the floating-point number calculation.

The probability of the particle to be inside the given x-boundaries is treated
as 100\%, so the probablity outside has to be 0. Therefore the problem has
aquivalent additonal infinite high potential walls at ‘xmin’ and ‘xmax’.
In case of non-decreasing wavefunction amplitudes outside the potential
boundaries, reconsideration of the calculated solutions is recommended.


\renewcommand{\indexname}{Python Module Index}
\begin{sphinxtheindex}
\def\bigletter#1{{\Large\sffamily#1}\nopagebreak\vspace{1mm}}
\bigletter{s}
\item {\sphinxstyleindexentry{schrodinger\_io}}\sphinxstyleindexpageref{index:\detokenize{module-schrodinger_io}}
\item {\sphinxstyleindexentry{schrodinger\_solver}}\sphinxstyleindexpageref{index:\detokenize{module-schrodinger_solver}}
\item {\sphinxstyleindexentry{schrodinger\_visualize}}\sphinxstyleindexpageref{index:\detokenize{module-schrodinger_visualize}}
\indexspace
\bigletter{t}
\item {\sphinxstyleindexentry{test\_schrodinger\_solver}}\sphinxstyleindexpageref{index:\detokenize{module-test_schrodinger_solver}}
\end{sphinxtheindex}

\renewcommand{\indexname}{Index}
\printindex
\end{document}